\chapter{Systematic Uncertainties in the Short Baseline Neutrino Program}

In the previous chapter, the motivation for the Fermilab Short Baseline Neutrino program was presented and the expected event rates were shown, as well as the methods of calculating an expected signal from a 3+1 model.  However, the most detailed simulation (or data analysis, for that matter) is not consequential without a robust calculation of systematic uncertainties.

In this chapter, the systematic uncertainties for the Short Baseline are discussed.  Of particular importance are the uncertainties from the flux and neutrino interactions.  The flux for the Booster Neutrino Beam, while among the best known neutrino beam fluxes, still has residual uncertainties of up to 15\% \cite{miniboone_flux_paper}.  Similarly, the uncertainty in the model of neutrino interactions has a 10 to 15\% normalization uncertainty for the quasi-elastic and resonant events that are most important to the oscillation searches.  Considering that the amplitude of any sterile neutrino oscillation effect is very small, with oscillation probabilities that peak at 1\% or less, constraining the systematic uncertainties in the Short Baseline Program is absolutely essential.

The strength of the Short Baseline Program's oscillation search comes, ultimately, from two factors:  the \lartpc technology allows excellent event identification and background rejections, and the near detector, SBND, allows for large cancellation of systematic uncertainties.  In this chapter, the method for quantifying the cancellation of systematic uncertainties is presented.


\section{General Framework for quantification of uncertainties}

In this analysis, the uncertainties that matter are the systematic uncertainties on the final distribution of event rates.  Since the goal is to produce a sensitivity calculation for an expected signal, the numerical value of the sensitivity can be calculated with a $\chi^2$ calculation:

\begin{equation}
\begin{centering}
\chi^2(\Delta m^2, \text{sin}^2 2 \theta ) = \sum_{i,j} [N^{null}_i - N^{osc}_i(\Delta m^2, \text{sin}^2 2 \theta ) ] \times E^{-1}_{i,j} \times [N^{null}_j - N^{osc}_j(\Delta m^2, \text{sin}^2 2 \theta ) ],
\end{centering}
\end{equation}

where $N^{null}_i$ is the expected event rate in the $i^{th}$ analysis bin with no oscillation signal, and $N^{osc}_i(\Delta m^2, \text{sin}^2 2 \theta )$ is the expected event rate in the $i^{th}$ analysis bin if there is an oscillation signal from a 3+1 model with the specified mass splitting and amplitude.  In the \nue appearance analysis, this is simplified to 
\begin{equation}
\begin{centering}
N^{null}_i - N^{osc}_i(\Delta m^2, \text{sin}^2 2 \theta ) = S_i(\Delta m^2, \text{sin}^2 2 \theta )
\end{centering}
\end{equation}
where S is the expected signal events from the specified parameters in the $i^{th}$ bin.

$E_{i,j}$ in the $\chi^2$ computation is the covariance matrix, a statistical tool to encode correlated uncertainties.  In practice, the computation of the covariance matrix is the most challenging aspect of the $chi^2$ calculation because it requires careful determination of how the uncertainties under study are correlated.  For this work, the correlations of uncertainties are quantified with the ``multiple universe'' method \footnote{Nothing to do with the cosmological idea of the multiverse}.  Much more will be said about the computation and use of the covariance matrix in Section~\ref{sec:covariance_matrix}.

\subsection{Multiple Universe Error Propagation and Reweighing methods}

In a complex chain of simulation and analysis such as a prediction of event rates in a neutrino detector, it can be challenging to understand the effect of, for example, an uncertainty of hadron production and the neutrino target on the final distribution of events in the detector.  Some intuitive knowledge is of course present: if the amount of neutrino producing particles at the target is under (or over) estimated, the event rates in the final analysis distribution at the detector will also be under (over) estimated.  To precisely quantify the relationship between initial variable underlying the simulation and the final distributions of events, a reweighing scheme with multiple universes is used.

\subsubsection{Reweighing Events}

As a concrete example, 



Multiverse analysis, drawing from physical distributions and propagating to the end distributions.  Reweighting method.

\section{Determination of Covariance Matrices}
\label{sec:covariance_matrix}

Using the methods described above for applying weights on an event-by-event basis, it's possible to generate a suite of ``Universes''
of event rate histograms, where the value of each analysis bin can be known in each universe as $N^i_{\text{Univ.} m}.$  In this document, since there are three detectors under consideration, the vector of event rates in each analysis bin, $N$, is a concatenation of the vector of event rates in each detector.  If there are $P$ total analysis bins in each detector, then 
\begin{equation}
\begin{centering}
\vec{N}_{\text{Nom.}} = \left(~N_{\text{Nom.}}^{1,~SBND},~\dots~N_{\text{Nom.}}^{P,~SBND},~N_{\text{Nom.}}^{1,~\uboone}~\dots~N_{\text{Nom.}}^{P,~\uboone},~N_{\text{Nom.}}^{1,~\icarus}~\dots~N_{\text{Nom.}}^{P,~\icarus} ~\right)
\end{centering}
\end{equation}
and in each universe where an underlying physical parameter has been varied:
\begin{equation}
\begin{centering}
\vec{N}_{\text{Univ.}~m} = \left(~N_{\text{Univ.}~m}^{1,~SBND},~\dots~N_{\text{Univ.}~m}^{P,~SBND},~N_{\text{Univ.}~m}^{1,~\uboone}~\dots~N_{\text{Univ.}~m}^{P,~\uboone},~N_{\text{Univ.}~m}^{1,~\icarus}~\dots~N_{\text{Univ.}~m}^{P,~\icarus} ~\right).
\end{centering}
\end{equation}

With these vectors, it's possible to calculate deviation from the nominal values due to the underlying uncertainties in an analysis bin:
\begin{equation*}
\begin{centering}
\sigma^i = \frac{1}{M}\sum_{\text{All Univ.}~m} \left( N^i_{\text{Nom.}} - N^i_{\text{Univ.~m}}\right)^2
\end{centering}
\end{equation*}

Covariance formula, etc. Correlation matrix, discussion of the meaning of the correlation matrix.  Correlations between detectors within the same event sample, also nue to numu correlations.

\section{Uncertainties from Neutrino Flux}

\label{section:flux_uncert}

Detailed discussion of the origins of the flux uncertainties.  The physical parameters that we have to constrain, and where the uncertainty ranges come from.  Dead-reckoned uncertainties in the event rates for each uncertainty, if possible, and dead-reckoned total uncertainty at each detector.

\section{Uncertainties from Neutrino Interactions}

\section{Residual Systematic Uncertainties}