\chapter{Liquid Argon Time Projection Chambers}

\section{Time Projection Chambers}

The Time Projection Chamber, abbreviated TPC, is a revolutionary particle detector concept first proposed in 1974 by David Nygren at Lawrence Berkeley National Lab.  Since then, the TPC has found applications in a broad array of particle physics experiments such as collider experiments at the LHC \cite{Lippmann:2104844,Aamodt:2008zz}, precision measurements of muon properties \cite{Luo:2015oca}, dark matter experiments \cite{Akerib:2012ys,Aprile:2011dd} and more.  The abundance of uses for the TPC technology stems from the versatile and robust ability of a TPC to track charged particles.

In general, a Time Projection Chamber is a volume filled with some neutral and inert material.  Commonly, noble gases and liquids are used though this is not required.  An electric field is applied to the entire medium, and in some cases a magnetic field is applied as well.  The electric field is generally applied by using a high voltage cathode as one surface of the detector.  The opposing surface, the anode, is typically instrumented with readout equipment.

Time Projection Chambers are designed to observe electrically charged particles.  In particular, a high energy charged particle (such as an electron, muon, pion, proton, etc.) can travel through the detector medium and will ionize the substance as it passes through, leaving a trail of electrons and ions.  The applied electric field, emanating from the cathode, serves to separate the ionization electrons from the ions and move the electrons towards the anode of the detector.  The drifted electrons form the basis of the measurement of the particle.  In particular, they appear as a projection of the original track onto the anode of the detector, and the distance from the anode is determined by the time it took for the electrons to drift.  Hence the name, Time Projection Chamber.  Far more detail is given in the description of liquid argon TPCs, below \ref{sec:argoneut_detector}

\section{History and Liquid Argon Time Projection Chamber Concepts}

The Liquid Argon Time Projection Chamber (LArTPC) was initially proposed in 1977 by Carlo Rubbia \cite{Rubbia:1977zz}, who later led the development of the ICARUS detector and R\&D programs.  At the time of it's concept, neutrino physics was dominated by bubble chamber detectors like Gargamelle, renowned for it's remarkable resolution of particle topologies.  Initially, the LArTPC was proposed as a way to combine high spatial resolution detectors with calorimetry measuring detectors in a way that is scalable to massive detectors.  As the field of neutrino physics approaches the largest LArTPC to date with DUNE \cite{DUNE}, it's worthwhile to recall the original advantages of the LArTPC technology as laid out in 1977 \cite{Rubbia:1977zz}:

\begin{itemize}

\item{\bf ``It is dense''} : The relative high density of liquid argon, at 1.4 $g/cm^3$, provides a sufficiently high neutrino interaction rate such that high statistics measurements are feasible.

\item{\bf ``It does not attach electrons and permits long drift times''}: As a long drift time is essential to large scale detectors to both maximize the mass of the detector and minimize the number of readout channels, the fact that argon itself does not attach free electrons is an essential ingredient to LArTPCs.

\item{\bf ``It has a high electron mobility''}: The high mobility makes drifting electrons from particle ionization in a short time a feasible task.

\item{\bf ``It is cheap''}:  A detector can not be scaled to massive sizes unless the fundamental building block of the detector is affordable.

\item{\bf ``It is easy to obtain and purify''}: Purification challenges have largely been overcome for LArTPCs.  In particular, the \uboone experiment has demonstrated a viable way to achieve high purity argon without purging the detector of impurities first.

\item{\bf ``It is inert and can be liquified with liquid nitrogen''}: This makes the cryogenic systems for LArTPCs reasonable to purchase and implement.

\end{itemize}


40 years after the original proposal, it is remarkable how relevant the initial advantages remain in the face of an experiment such as DUNE. 

Since the original proposal, some additional advantages of LArTPCs have been noted and are worth mentioning.  For example, the scintillation of Liquid Argon has been successfully characterized and is measurable in coincidence with the drift ionization.  For large detectors, especially surface detectors, this allows the ability to match scintillation light to ionization tracks to reject out of time events such as cosmic particles.  It also allows the implementation of a hardware based trigger to filter neutrino interactions online.  For even modest sized LArTPCs, this can be an essential aspect to control data rates and ease computing requirements.

\section{Design of LArTPCs}
\label{sec:argoneut_detector}

As mentioned above, the Liquid Argon TPC has a long history of development.  This section presents the details of a modern LArTPC in it's design, given in the context of the \argoneut detector. A comprehensive and detailed description of the \argoneut detector is given in \cite{Anderson:2012vc}.

\subsection{\argoneut Time Projection Chamber}

The \argoneut TPC is a rectangular volume of liquid argon that measures 40 cm high (Y direction), 47 cm wide (X direction), and 90 cm long (Z direction).  In total, this corresponds to about 170 liters of Liquid Argon.  In it's running configuration, neutrinos from Fermilab's NuMI beam \ref{sec:numi_beam} enter nearly parallel to the Z direction, with a slight downward direction.  On the left side of the detector in the beam direction is the high voltage cathode providing a uniform electric field of 500 V/cm throughout the TPC (corresponding to approximately -23 kV of voltage at the cathode).  Opposite the cathode is the anode, composed of three wire planes, of which only two are instrumented for readout.

%insert argoneut tpc picture.
\begin{figure}[h]
  \centering
  \includegraphics[width=0.75\textwidth]{lartpc_figures/argoneut_tpc_and_cryostat.pdf}
  \caption[The \argoneut TPC]{The \argoneut TPC positioned just outside of its cryostat.  The wire planes and the readout electronics are visible on the right side of the TPC.}
  \label{fig:argoneut_tpc}
\end{figure}

In the detector, as a neutrino interacts it produces outgoing particles, most commonly: muons, protons, neutrons, pions (charged and neutral), photons and electrons.  Naturally, the possible particles produced in a neutrino interaction is much broader than this short list, but this comprises the most frequent particles.  In the case of the electrically charged particles, the particle will ionize the argon atoms as it moves through the detector.  The ionization produced is a statistical quantity, but the average expected ionization depends strongly on the momentum and mass of the particle in question.  In general, particles with higher mass and lower momentum produce larger ionization per unit distance traveled \cite{bethe-bloch}.  The ionization per unit distance, measured most frequently in the units MeV/cm, is a very powerful tool for calorimetric identification of particles (as demonstrated in Chapter \ref{chp:electrons}.

Neutral particles, such as neutrons and photons, do not ionize the argon atoms as they traverse the detector.  However, these particles can still interact with the argon and produce charged particles visible to the TPC instrumentation.  Neutrons frequently will scatter off of an argon nucleus and produce a recoiling proton, which can be observed in the detector.  Photons can produce electromagnetic showers through Compton scattering and pair production, described more fully in Chapter \ref{chp:electons}.

After the particles from the neutrino interaction have produced ionization in the detector, the electric field separates the ions and electrons from each other.  Naturally, the separation is imperfect and depends on the strength of the electric field, the amount of ionization, as well as the angle of ionization with respect to the field.  This effect, known as recombination of electrons and ions, has been studied in detail in the \argoneut detector \cite{Acciarri:2013met}.  In general, this effect causes a quenching of the observed electrons compared to the true ionizing power of the high energy particles as seen in Figure \ref{fig:argoneut_recombination}.

The uniformity of the electric field in the \argoneut detector is maintained with a field shaping system of electrodes.  The electrodes are plated on  to the interior surface of the volume between the cathode and anode, and are held at a voltage linearly decreasing from cathode to anode.  In \argoneut, the field shaping strips are 1cm wide and seperated by 1cm, and there are 23 strips total.  This technique, however, is utilized in a variety of TPC experiments.

%insert dQ/dx vs dE/dx plot here.
\begin{figure}[h]
  \centering
  \includegraphics[width=0.5\textwidth]{lartpc_figures/recombination_dqdx_dedx.pdf}
  \caption[\argoneut Recombination]{Measurement of the recombination effect in \argoneut using stopping protons, at 500 V/cm. \cite{Acciarri:2013met}}
  \label{fig:argoneut_recombination}
\end{figure}

Once the electrons have been separated from the ions, they drift towards the readout wires of the TPC.  Though argon itself does not attach electrons, impurities in the argon can do so.  The amount of drifting electrons declines as a function of the distance it has to drift.  This decline is well modeled with an exponential decline, and the decay constant is referred to as the electron ``lifetime.'' Proper calorimetry must take the lifetime of the electrons into account on hit by hit basis to correctly account for the effect of the impurities in the liquid argon.  In \argoneut, the electron lifetime is measured in data by comparing the amplitude of this at different drift distances as seen in Figure \ref{fig:argoneut_lifetime}.

%insert lifetime plot here.
\begin{figure}[h]
  \centering
  \includegraphics[width=0.5\textwidth]{lartpc_figures/argoneut_lifetime.pdf}
  \caption[\argoneut Lifetime Measurement]{The electron lifetime in \argoneut is computed run by run empirically, using a sample of depositions in the TPC from minimally ionizing particles.  Shown here is the fit, using an exponential, for run 648 giving an electron lifetime of 742 $\pm \mu s$ (statistical error only).}
  \label{fig:argoneut_lifetime}
\end{figure}


As alluded to above, the \argoneut detector has three planes of wires at the anode, two of which are instrumented.  The first plane, composed of 225 wires oriented vertically, serves as a shielding plane for the other wires and to provide shaping to the electric field through the TPC.  The three planes are spaced with 4mm between each other.  The second plane, referred to as the ``induction plane,'' contains wires that are set at +60$^o$ to the beam axis.  As electrons cross the shield plane, they approach the induction plane wires.  The wires are biased, however, such that the electrons drift around the individual wires.  The approaching and subsequent passing of electrons induces a current on these wires (hence the name ``induction plane'') and bipolar pulse shape is recorded by the readout electronics for wires that observe electrons.  See figure \ref{fig:argoneut_signals} for examples of this pulse.

The final set of wires, dubbed the ``collection plane,'' is biased such that it collects the drifting electrons onto it and they are observed as a pulse of charge by the electronics system.  The collection plane is set at an angle of -60$^o$ to the beam direction.  The two instrumented planes each have wire spacings of 4mm, and sample at 5.05 MHz.  In total, the instrumented planes have 240 wires in each plane.  Naturally, since the wires are at an angle with respect to the TPC axes, not all wires are of the same length.  Most wires, 144 of 240 in each plane, are 46.2 cm long.  The shortest wires are 3.7 cm long.

The sense wires are readout with a system of electronics sampled every 198 ns, and the readout system has a sensitivty of 7.49 ADC/fC of charge recorded.  This gives a signal to noise ratio of 15 or higher for minimally ionizing particles in the TPC.  An in depth description of the \argoneut readout electronics is available in \cite{Anderson:2012vc}


%insert wire pulse figure
\begin{figure}[h]
  \centering
  \includegraphics[width=\textwidth]{lartpc_figures/argoneut_signal.pdf}
  \caption[Deconvolution of \argoneut Signals]{Raw and deconvoluted signal shapes from the \argoneut detector.  On the top is shown the induction pulse.  The bipolar shape of the pulse in the induction plane is corrected during the deconvolution stage.  On both planes, a Gaussian hit fitting technique is used to determine the amount of charge recorded. Figure from \cite{Anderson:2012vc}.}
  \label{fig:argoneut_signals}
\end{figure}

\section{\label{sec:lartpc_reconstruction} Event Imaging and Reconstruction}

Each wire measures a signal of electrons as they drift, as a function of time.  When the wires are arrayed in an image in sequential order, such that the x axis is wire number and the y axis is time tick, 2D images are formed such as in figure \ref{fig:argoneut_data}.  As seen in figure \ref{fig:argoneut_projection}, the wire planes represent projections of the 3D data onto a plane that is orthogonal to the wires themselves.

%insert argoneut data picture.

%insert argoneut projection picture
\begin{figure}[h]
  \centering
  \includegraphics[width=\textwidth]{lartpc_figures/argoneut_plane_projection.pdf}
  \caption[\argoneut 3D projection]{Representation of the projection of the LArTPC in \argoneut.  The wire and time axes give a 2D image that represents a projection of the 3D charge depositions on to the 2D surfaces shown in blue.  Figure from \cite{Anderson:2012vc}.}
  \label{fig:argoneut_projection}
\end{figure}

\subsection{Deconvolution}
The reconstruction of these images into a 3D event starts at the lowest level, filtering and deconvolution of the wire signals.  In general, the number of electrons recorded by a given wire as a function of time is not prefectly matched by the ADC signals read out by the detector.

To correct for this, a deconvolution process is applied to each wire.  As seen in figure \ref{fig:argoneut_signal_shaping}, the response of the detector to a delta function introduces a spread of signal which is removed using a scheme with the Fast Fourier Transform. The response of each channel is measured with external pulse generators.  The convolution theorem then allows the removal of the detector response by taking the  inverse Fourier transform of $\frac{v[t]}{r[t]}$, where $v[t]$ is the Fourier transform of the recorded waveform and $r[t]$ is the Fourier transform of the channel's response.  Figure \ref{fig:argoneut_deconvolution} shows the result of applying deconvolution to \argoneut data in the collection and induction planes.  In addition, the deconvolution for the induction plane removes the bipolar behavior to make hit finding easier.

% need an image of argoneut deconvolution

\begin{figure}[h]
  \centering
  \includegraphics[width=\textwidth]{lartpc_figures/argoneut_signal_shaping.pdf}
  \caption[Signal Shaping in \argoneut]{On the left, an image of the idealized detector response to drift electrons in the induction and collection plane.  On the right, the response of the electrons filter and digitization to a delta function pulse.  Figure from \cite{Anderson:2012vc}.}
  \label{fig:signal_shaping}
\end{figure}

\subsection{Hit Finding}

For each wire in the detector, a hit finding algorithm is used to locate the regions of the readout with electron deposition signals.  While there are several different hit finding algorithms available in LArSoft \cite{Church:2013hea}, the official LArTPC reconstruction software, the all follow a generalized procedure.

First, a deconvolved (and noise filtered) wire signal is scanned for regions of signal above a specified threshold.  Often, the baseline threshold of hit finding depends on whether the signal is from collection or induction planes.

Next, the regions of interest are fitted with an analytic function to allow a precise determination of the time tick, peak, and integral of the charge deposited.  The most common function used is a Gaussian.  In some cases, and commonly in neutrino interactions, hits that are close to each other from different particles will have overlapping regions.  In this case, the multiplicity of the region above threshold can be determined to help tracking algorithms accurately distribute hits between different particles.  An example of this is seen in Figure \ref{fig:argoneut_hit_multiplicity}.  In general, complicated regions with multiple hits are fit with several Gaussians summed together.

\begin{figure}[h]
  \centering
  \includegraphics[width=\textwidth]{lartpc_figures/argoneut_hit_multiplicity.pdf}
  \caption[Hit Finding in \argoneut]{A neutrino vertex as seen in the induction view in \argoneut.  The top left shows the reconstructed signals above threshold.  The other figures show the wire signal moving away from the vertex: the initial signal is wider than normal, and as the tracks diverge in the detector the two peaks are resolved.  Figure from \cite{Anderson:2012vc}.}
  \label{fig:argoneut_hit_multiplicity}
\end{figure}

\subsection{Cluster, Tracking and 3D Reconstruction}

Once the wire signals have been deconvolved, and the signal depositions have been reconstructed as hits, a number of higher level steps remain between hits and physics data.  First, hits must be grouped together based on which particle they originated from.  In general, this is an extremely difficult problem with no simple answer.  For particles like muons and protons, which produce simple tracks of hits in the detector, it is not impossible and a lot of progress and achievements have been made.  For more complicated events, such as electromagnetic showers, clustering remains the weakest point of the reconstruction chain.

For a track like particle, in general, the groups of hits are associated together into clusters.  These cluster are then matched across the planes of the detector (two planes in argoneut, but many state of the art detectors have 3).  Though the planes offer different projections of the 3D events into 2D, the drift direction (time tick direction in \ref{fig:argoneut_data}) is a common axis in every projection.  Therefore, the most useful metric to determine if two clusters are from the same track in the argon is the time if took those clusters to drift to the wires.

Once clusters from multiple planes have been matched together, the wire information between the two clusters can be used to determine where in the Y-Z plane the clusters overlap.  This is because each wire intersects the other plane's wires at most once, so if a charge deposition from one plane is matched to one on another plane, it uniquely determines the location of the 3D charge (The X coordinate comes from the drift time).

Almost all of the details of 3D tracking and reconstruction have been skimmed here, as they are not crucial to the work presented in this thesis.  However, a great detail of knowledge and techniques is reported in many references \cite{reco:tracking_ref} \cite{reco:other_refs.}

\subsection{\label{subsec:lartpc_calibration} Calibration}

For a \lartpc to perform physics studies with calorimetric information, it is essential to accuratly calibrate the detector response to charge depositions on the wires.  For \argoneut, this was performed with large sample of crossing muons as reported in \cite{Anderson:2012mra}.  That analysis demonstrated that muons induced from upstream interactions (known as ``through-going muons'' in \argoneut) can be used as a known source of  ionization in the detector, as shown in Figure~\ref{fig:mpv_muons}.  For \argoneut, the mean momentum of the through-going muons was estimated at 7 GeV/c.

\begin{figure}[tb]
  \centering
  \includegraphics[width=\textwidth]{lartpc_figures/mpv_muons.png}
  \caption[Most Probable Ionization, Muons]{Most probable ionization amounts for muons traversing liquid Argon, as a function of momentum.}
  \label{fig:mpv_muons}
\end{figure}

To calibrate the detector from a sample of muons, the dE/dx of each deposition measured by the wires of the TPC can be collected into a histogram and the shape is fit with a Gaussian-convolved Landau distribution, as demonstrated in Figure~\ref{fig:coll_mpv_muons}.  Unlike \cite{Anderson:2012mra}, however, there are two differences performed for the calibration used in the analyses described in Chapters~\ref{chp:emshowers} and \ref{chp:nue_xsec}: the calibration constants are calculated on a wire-by-wire basis, instead of for the entire detector.  Second, the calibration is calculated for the collection and the induction plane, instead of just the collection plane.  Due to advancements in deconvolution and hit finding since the original publication of the \argoneut calibrations, the induction plane can now be shown to be a useable plane for calorimetry, as seen in Figure~\ref{fig:coll_ind_differences}.  The two planes show agreement in the calorimetric values calculated for the crossing muons.  In the end, the calibration constants are determined to be 36.4 $\pm$ 2.48 [fC/(ADC*tick)] for the collection plane, and 143 $\pm$ 10.3 [fC/(ADC*tick)].

\begin{figure}[tb]
  \centering
  \includegraphics[width=0.45\textwidth]{lartpc_figures/collection_muons_total.png}
  \includegraphics[width=0.45\textwidth]{lartpc_figures/collection_muons_total_sim.png}
  \caption[Muon dE/dx Distributions]{Most probable ionization amounts for muons traversing liquid Argon, as a function of momentum.}
  \label{fig:coll_mpv_muons}
\end{figure}

\begin{figure}[p]
  \centering
  \includegraphics[width=0.45\textwidth]{lartpc_figures/mean_coll_vs_ind_sim_muons.png}
  \includegraphics[width=0.45\textwidth]{lartpc_figures/mean_coll_vs_ind_data.png}
  \includegraphics[width=0.45\textwidth]{lartpc_figures/median_coll_vs_ind_sim_muons.png}
  \includegraphics[width=0.45\textwidth]{lartpc_figures/median_coll_vs_ind_data.png}
  \includegraphics[width=0.45\textwidth]{lartpc_figures/mpv_coll_vs_ind_sim_muons.png}
  \includegraphics[width=0.45\textwidth]{lartpc_figures/mpv_coll_vs_ind_data.png}
  \caption[Cross Plane Calibration Checks]{A comparison of the mean, median, and most probable value for crossing muons between the collection (x axis) and induction (y axis) planes.  Simulation is shown on the left, data on the right.  There is good agreement between collection and induction planes, as well as between simulation and data.}
  \label{fig:coll_ind_differences}
\end{figure}

\subsection{Particle Identification and Calorimetry}

In a LArTPC, the calorimetric identification of particles is based upon the behavior of charged particles moving through the argon.  The energy deposited per centimeter is dictated by the Bethe-Bloch equations, and the properties in argon of common particles are seen in Figure \ref{fig:bethe_bloch}.

As a particle loses energy, it's amount of ionization decreases until it reaches a minimum before the ionization spikes to very high values.  Due to the limited resolution of the detector, however, the observed dE/dx values for a given particle will increase as the particle comes to a rest.  As seen in Figure \ref{fig:residual_range}, this measure of dE/dx versus residual range allows calorimetric separation of particles.  In particular, protons are easily separated from muons and pions with this measure.

\begin{figure}[tb]
  \centering
  \includegraphics[]{lartpc_figures/residual_range.pdf}
  \caption[Residual Range]{dE/dx versus residual range of various particles in liquid Argon.  The values of dE/dx vs. Residual range for a particle observed in the detector can be used to identify the particle's type.}
  \label{fig:residual_range}
\end{figure}

Since LArTPCs also offer bubble chamber quality images, the topology of an event can give excellent ways to distinguish particles.  As seen in Figure \ref{fig:mu_pi}, particles like muons and pions that are difficult to distinguish with calorimetry can often be separated based on subsequent interactions within the TPC.



\subsection{MINOS}

\argoneut is fortunate in that it was located directly upstream of the MINOS near detector, which is a magnetized tracking detector \cite{MINOS}.  This gives \argoneut a distinct trait that no other LArTPC has had: muon sign selection for muons produced in \argoneut that enter the MINOS near detector.

\begin{figure}[h]
  \centering
  \includegraphics[width=\textwidth]{lartpc_figures/minos.pdf}
  \caption[\argoneut and MINOS]{An event display depicting the \argoneut experiment and the MINOS near detector. \argoneut is the small box in the foreground.  The tracks represent TPC data of \numu CC interactions that were successfully tracked and matched into the MINOS near detector.  Figure from \cite{Anderson:2012vc}.}
  \label{fig:signal_shaping}
\end{figure}

\argoneut is only 90cm long at it's longest dimension, and since the NuMI beam has neutrino energies of 10+ GeV, it is extremely rare for muons produced in \argoneut to stop within the detector.  This enabled several precision measurements of muon neutrino cross sections on argon by looking for neutrinos that interact in \argoneut, and tracking them through the MINOS near detector \cite{Anderson:2011ce}, \cite{Acciarri:2014isz}.

For the analyses presented in this thesis, MINOS is not used as a muon spectrometer directly.  Instead, since the target interaction is electron neutrinos, MINOS is able to provide rejection of muon neutrino events.

\section{\label{sec:microboone} \uboone}

\section{\label{sec:future_tpcs} Future LArTPCs}


% \uboone is the largest \lartpc in the United States, and the second largest in the world.  Originally proposed in 2007, it was designed to confirm or rule out the \MB low energy excess.

% Details of the \uboone detector.  PMT systems, low energy excess search.

% \subsection{Future \lartpcs}

\subsubsection{SBND}

% Details of SBND, PMTS, proximity to beam, flux window, etc.

\subsubsection{\icarus}

% More of the same

