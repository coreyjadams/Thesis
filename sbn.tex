\chapter{Short Baseline Neutrino Program}

This chapter will describe the Short Baseline Neutrino Program, and the anomalies it is seeking to resolve.

\section{Motivation and Goals}

Over the past two decades, experimental hints of beyond the Standard Model neutrino physics have appeared in several distinct subfields of neutrino experiments.  Taken together, the constitute hints at oscillation physics on mass splitting scale that is inconsistent with the known oscillations models.  This section will briefly summarize the current anomalies in this area of neutrino oscillations, known as Short Baseline oscillation physics.

\subsection{LSND}

\subsection{Reactor Experiments}

\subsection{Source Experiments}

\subsection{\MB}

\subsection{Global Fits}

\section{FermiLab's Short Baseline Neutrino Program}

To address the anomalies described above, Fermilab is pursuing a program of short baseline neutrino experiments along it's Booster Neutrino Beam.  The first experiment, \uboone, started operations in 2015 and is designed to resolve the \MB anomaly.  Subsequently, two other detectors will join \uboone along the Booster Beam at 100m and 660m.  This section will describe the detectors, their design goals, 

\section{\uboone}

\uboone is the largest \lartpc built in the United States to date, and the second largest in the world.  Originally proposed in 2007, it was designed to not just 

Details of the \uboone detector.  PMT systems, low energy excess search.

\section{Future \lartpcs}

\subsection{SBND}

Details of SBND, PMTS, proximity to beam, flux window, etc.

\subsection{\icarus}

More of the same

\section{Physics Program}

What physics the SBN can do: sterile searches, try to rule out things, precision cross sections.

\section{Simulation and Monte Carlo Predictions of Event rates}

How to simulate the events in the detectors, the flux and cross section models.

\subsection{Simulated event reconstruction}

Smearing and other attempts to mimic real reconstruction.  Electron photon separation, cosmics and dirt events for completeness.  Event rate distributions.
