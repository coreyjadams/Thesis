\chapter{Conclusions} 

Neutrino physics has grown substantially since the initial discovery of the neutrino, especially with the conclusive evidence of neutrino oscillations and neutrino mass, for which the Nobel prize was awarded in 2015.  It is entering an era where deviations of even a few percent from the expected model of neutrino oscillations and interactions are detectable.  The next generation of neutrino detector technology is already running: fine granularity tracking detectors such as liquid argon time projection chambers are the preferred choice of neutrino experiments in the 1 GeV range.

In this thesis, a study of the expected backgrounds to the SBN Program were presented, and an expectation of the expected event rates including an estimate of the signal from a 3+1 sterile neutrino oscillation was shown.  The SBN Program, however, will require a measurement of electron neutrinos to extraordinary precision, and so a detailed understanding of the systematic uncertainties of its measurements are necessary.  In Chapter~\ref{chp:systematics}, it was shown that the uncertainties from the neutrino flux and cross sections can be constrained to several percent uncertainty by exploiting a multi-detector analysis.

The estimate of systematic uncertainties for the SBN Program did not include the uncertainties from sources such as detector effects or reconstruction efficiencies.  However, the strength of the SBN Program lies in the fact that its three detectors are the same technology along the same neutrino beam, and the systematic uncertainties largely cancel.  The uncertainties associated with detector effects and event reconstruction will need to be studied before the final analysis of the SBN program. However, if history is our guide, like other multi-detector experiments with precision detectors, the SBN Program will make a substantial impact on neutrino physics.

The first step toward the physics goals of the SBN Program, as well as DUNE and \uboone, is the observation and characterization of its signature oscillation channel.  In Chapter~\ref{chp:emshowers} the first observation of low energy electron neutrinos in a liquid argon time projection chamber was presented.  This measurement will be extended in the future to an electron neutrino cross section on argon, after the development of automated event selection for electromagnetic showers.  The critical next step is the development of an automated event selection for electron neutrinos in liquid argon.

The measurements of oscillations and electron neutrino appearance in many experiments will be constrained by its level of background rejection from high energy photons in the energy range of hundreds of MeV.  Though liquid argon time projection chambers have long promised exquisite rejection of high energy gammas, this work is the first demonstration of those abilities with data.  It was found that a purely topological selection of electron neutrinos could produce a sample that was 80 $\pm$ 15 \% pure, and a calorimetric cut can be very efficiently applied to an electron neutrino sample to further reduce backgrounds from high energy photons.  Future analyses can apply both of these results to measure electron neutrinos in the 1 GeV range with excellent purity.