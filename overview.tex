\chapter{Summary}

This thesis presents a detailed characterization of electron neutrinos in a state of the art neutrino detector, the liquid argon time projection chamber.  The signature of electron neutrinos in LArTPCs is critical to the US accelerator based neutrino physics program, at both short and long baselines.  Chapters \ref{chp:intro}, \ref{chp:beams}, and \ref{chp:lartpcs} present an overview of current neutrino physics, including how the field of neutrino physics reached its current state, as well as a description of both the neutrino beams and detector technologies needed to advance the field further.  Chapter \ref{chp:sbn} presents the current short baseline anomalies that hint towards non standard neutrino oscillations and the experimental outlook of the Fermilab Short Baseline Neutrino Program, while Chapter \ref{chp:systematics} highlights the importance of carefully studying and accounting for the uncertainties in Fermilab's program.  Finally, \ref{chp:emshowers} presents the first experimental observation of electron neutrinos in the 1 to 10 GeV range in a liquid argon time projection chamber, laying the groundwork for a decade's worth of precision neutrino measurements.