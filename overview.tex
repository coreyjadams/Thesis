

This thesis presents several interconnected topics in the field of neutrino physics, relevant to the state of the art detection technology, liquid argon time projection chambers.  Chapters \ref{chp:intro}, \ref{chp:beams}, and \ref{chp:lartpcs} present an overview of current neutrino physics, including some how the field of neutrino physics reached its current state, as well as a detailed description of both the neutrino beams and detector technologies needed to advance the field further.  Chapter \ref{chp:sbn} presents the current short baseline anomalies that hint towards non standard neutrino oscillations and the experimental outlook of the Fermilab Short Baseline Neutrino Program, while Chapter \ref{systematics} highlights the importance of carefully studying and accounting for the uncertainties in Fermilab's program.  Finally, \ref{chp:emshowers} presents the first experimental observation of low energy electron neutrinos in a liquid argon time projection chamber, laying the groundwork for a decades worth of precision neutrino measurements.