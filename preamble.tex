\usepackage{geometry} % you need this for yalephd.cls to work.
\usepackage{graphicx} % you probably want the rest of these.
\usepackage{dcolumn}
\usepackage{bm}
\usepackage{amsmath}
\usepackage{amsfonts}
\usepackage{amssymb}
\usepackage{appendix}
\usepackage{comment}
\usepackage{cite}
\usepackage{notoccite}
\usepackage{xspace}
\usepackage{subcaption}
\usepackage[makeroom]{cancel}
\usepackage{adjustbox}
\usepackage{grffile}
\usepackage{braket}
\usepackage{placeins}
\usepackage{tabularx}
\usepackage{rotating}
% This is a font, kind of similar to Times New Roman
\usepackage{mathptmx}

\usepackage{fancyhdr}
\pagestyle{fancy}
\fancyhf{}
\rfoot{Page \thepage}
 

\renewcommand{\chaptermark}[1]{\markboth{#1}{}}

\fancyhead[R]{%
   % We want italics
   \itshape
   % The chapter number only if it's greater than 0
   \ifnum\value{chapter}>0 \chaptername\ \thechapter \hfill  \fi
   % The chapter title
   \leftmark}
\fancyfoot[C]{\thepage}



\usepackage{hyperref}
\hypersetup{
    colorlinks,
    citecolor=black,
    filecolor=black,
    linkcolor=black,
    urlcolor=black
}

%% command definitions
\newcommand{\lartpc}{\text{LAr-TPC}\xspace}
\newcommand{\lartpcs}{\text{LAr-TPCs}\xspace}
\newcommand{\uboone}{\text{MicroBooNE}\xspace}
\newcommand{\MB}{MiniBooNE\xspace}
\newcommand{\uB}{MicroBooNE\xspace}
\newcommand{\SB}{SciBooNE\xspace}
\newcommand{\sbnd}{SBND\xspace}
\newcommand{\icarus}{ICARUS-T600\xspace}
\newcommand{\nova}{\text{NO\ensuremath{\nu}A}\xspace}
\newcommand{\argoneut}{\text{ArgoNeuT}\xspace}

\newcommand{\numu}{\ensuremath{\nu_{\mu}}\xspace}
\newcommand{\nue}{\ensuremath{\nu_{e}}\xspace}
\newcommand{\nutau}{\ensuremath{\nu_{\tau}}\xspace}
\newcommand{\numubar}{\ensuremath{\bar\nu_{\mu}}\xspace}
\newcommand{\nuebar}{\ensuremath{\bar\nu_{e}}\xspace}
\newcommand{\nutaubar}{\ensuremath{\bar\nu_{\tau}}\xspace}
\newcommand{\nuone}{\ensuremath{\nu_{1}}\xspace}
\newcommand{\nutwo}{\ensuremath{\nu_{2}}\xspace}
\newcommand{\nuthree}{\ensuremath{\nu_{3}}\xspace}
\newcommand{\nufour}{\ensuremath{\nu_{4}}\xspace}
\newcommand{\nufive}{\ensuremath{\nu_{5}}\xspace}
\newcommand{\nualpha}{\ensuremath{\nu_{\alpha}}\xspace}
\newcommand{\nubeta}{\ensuremath{\nu_{\beta}}\xspace}
\newcommand{\pizero}{\ensuremath{\pi^{0}}\xspace}

\newcommand{\numunue}{\ensuremath{\numu \rightarrow \nue}\xspace}
\newcommand{\numunuebar}{\ensuremath{\bar{\nu}_{\mu} \rightarrow \bar{\nu}_{e}}\xspace}
\newcommand{\numunueboth}{\ensuremath{\overset{\text{\tiny (}-\text{\tiny )}}{\numu} \rightarrow \overset{\text{\tiny (}-\text{\tiny )}}{\nue}\xspace}}
\newcommand{\numudis}{\ensuremath{\numu \rightarrow \nu_{x}}\xspace}
\newcommand{\nuedis}{\ensuremath{\nue \rightarrow \nu_{x}}\xspace}
\newcommand{\numunumu}{\ensuremath{\numu \rightarrow \numu}\xspace}
\newcommand{\dmsq}{\ensuremath{\Delta m^2}\xspace}
\newcommand{\dmsqfourone}{\ensuremath{\Delta m^2_{41}}\xspace}
\newcommand{\sinth}{\ensuremath{\sin^22\theta}\xspace}
\newcommand{\Enu}{\ensuremath{E_{\nu}}\xspace}
\newcommand{\dcp}{\ensuremath{\delta_{\mathrm{CP}}}}
\newcommand{\lenu}{\ensuremath{L/E_{\nu}}\xspace}

\newcommand{\GeV}{\ensuremath{\mbox{GeV}}\xspace}
\newcommand{\MeV}{\ensuremath{\mbox{MeV}}\xspace}
\newcommand{\GeVc}{\ensuremath{\mbox{GeV}/c}\xspace}
\newcommand{\MeVc}{\ensuremath{\mbox{MeV}/c}\xspace}
\newcommand{\GeVcc}{\ensuremath{\mbox{GeV}/c^2}\xspace}
\newcommand{\eVsq}{\ensuremath{\mbox{eV}^2}\xspace}


% Bibliography stuff:
\bibliographystyle{unsrt}
